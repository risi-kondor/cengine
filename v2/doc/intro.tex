\Cengine{} is a lightweight compute engine designed to %run in the background, below user level \cpp{} code.  
%The puropose of \Cengine{} is to 
parallelize numerical computations at run time by 
%dynamically parallelizes %unstructured 
%numerical computations in a combination of two ways: 
%mathematical workloads by 
\begin{compactenum}[~~(a)]
\item distributing computation across multiple CPU threads and/or  
\item batching together operations of the same kind for parallel execution on the GPU. 
\end{compactenum}
\Cengine{} employs the delayed execution model of computation to decouple user code %calling code 
from worker threads, 
%give itself freedom in which 
%order to execute the operations demanded by user level code. 
%Internally, it 
and maintains an internal dependency graph %between operations   
to ensure correctness. %that dependency relationships are not violated. 

From the user code side, the engine expects a sequence of simple instructions. For example, 

\texttt{~~~c=engine.push<ctensor\_add\_op>(a,b)}

tells the engine to add tensors \ccode{a} and \ccode{b} and store the result in \ccode{c}. 
Instead of executing this instruction directly, 
%The instructions are not executed immediately. 
%Instead, 
\Cengine{} queues the corresponding \ccode{ctensor\_add\_op} internally, %operator to its internal queue, 
and executes it later, when either a CPU threads becomes 
available or a sufficient number of operations of the same type have accumulated to 
make executing them on the GPU in batch economical. 
%The result of a sequence of operations is returned to the user once it is ready. 

\Cengine{} offers a small collection of built-in data types %such as \ccode{rscalar, cscalar} and \ccode{ctensor} 
to represent real/complex valued scalar and tensor objects, and a corresponding complement of 
basic arithmetic and linear algebra operators. 
However, the engine is primarily designed to be used with user defined classes and operators. 
%designed to manage user defined data types and 
%operators just as well as it can manage its built-in types. 
%, and in fact we expect that in most use cases the engine will be extended in this way. 

\Cengine{} is written in standard \cppe{} and requires no other libraries besides the standard 
template library and CUDA/CUBLAS for GPU functionality.  

