\Cengine{} is a header-only library that does not require separately compiling any object files. 
Assuming that the library's root directory is \ccode{\$(CENGINE\_ROOT)}, the core header files 
are located in \ccode{\$(CENGINE\_ROOT)/include} and \ccode{\$(CENGINE\_ROOT)/engine}. 
In addition, if the user wishes to use the built in scalar and tensor classes, they must also 
include \ccode{\$(CENGINE\_ROOT)/backend/scalar} and \ccode{\$(CENGINE\_ROOT)/backend/tensor}. 
Thus, to compile your own code \ccode{foo.cpp} with \Cengine{} the compiler might be invoked as for example 

\texttt{~~~c++ -o foo foo.cpp -std=c++11 -lstdc++11 -lm -I\$(CENGINE\_ROOT)/include\\ 
\mbox{}~~~-I\$(CENGINE\_ROOT)/engine 
\$(CENGINE\_ROOT)/backend/scalar \$(CENGINE\_ROOT)/backend/tensor}. 

Some systems require also linking the pthreads library as \ccode{-lpthreads}. 

The top level executable of any code compiled with \Cengine{} must include \ccode{Cengine.cpp}, which 
defines certain global variables and starts the compute engine. 

\subsection*{Compile time flags and variables}
\addcontentsline{toc}{section}{Compile time flags and variables}

Use \ccode{\#define} to set any of the following compile time flags. 

\begin{tabularx}{\textwidth}{|l|l|X|}
\hline
Flag&default&\\
\hline
\hline 
DEBUG\_ENGINE\_FLAG & off & When set, the engine will print detailed diagnostic information as it runs. \\
\hline 
\end{tabularx}
\bigskip

Use \ccode{\#define} to set the value of any of the following compile time variables. 

\begin{tabularx}{\textwidth}{|l|l|X|}
\hline
Flag&default&\\
\hline
\hline 
CENGINE\_NWORKERS& 4 & Default number of CPU worker threads.\\
\hline 
\end{tabularx}