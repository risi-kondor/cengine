\begin{cldescr}{\ccode{BatchedOperator}}

\ccode{BatchedOperator} is the virtual base class of all \ccode{Coperator}s that are batchable. 
New batchable operators are defined by subclassing \ccode{BtachedOperator}. 

\parentclass{Coperator}{}

\begin{clgroup}[Member variables]
\member{Cnode* owner}{Pointer to the compute graph node associated with this operation.}
\member{vector<Cnode*> inputs}{Vector of pointes to the compute graph nodes corresponding to the inputs 
of this operation.}
\end{clgroup}

\begin{clgroup}[Methods]
\member{virtual int batcher\_id() const=0}{Return the index of this operator class. 
The index is a static variable of the operator class that is set by the engine itself.}
\member{virtual void set\_batcher\_id(const int i=0}{Set the index of this operator class to \m{i}. 
This function is used by the engine to set the index of the operator class, the first time the 
operator is encountered.}
\\ 
\member{virtual void exec()=0}{
As in the \ccode{Coperator} base class, this function defines the operator's operation when 
executed on data objects individually (not batched).} 
\member{virtual void batched_exec()=0}{
This function defines the operator's operation when executed on a batch of inputs. 
}
\\
\member{SIGNATURE signature() const=0}{
Return a \ccode{SIGNATURE} object that captures the given operator's signature.}
\member{Batcher* spawn\_batcher() const=0}{
Create a new \ccode{Batcher} object for this operator.  
}
\end{clgroup}
\\
\begin{clgroup}[I/O]
\member{virtual string str() const=0}{
Return a human readable representation of the operator for debugging purposes. 
}
\end{clgroup}


\end{cldescr}