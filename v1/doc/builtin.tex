\Cengine{} provides three built-in ``virtual'' data types:
\begin{compactenum}[~~\m{\circ}]
\item \ccode{rscalar} to represent real valued scalars;
\item \ccode{cscalar} to represent complex valued scalars;
\item \ccode{ctensor} to represent complex valued tensors and matrices. 
\end{compactenum} 
All three use single precision arithmetic and support parallelism through bundles. 
The types are ``virtual'' in the sense that corresponding can only be created and 
accessed by pushing the appropriate operators to the \ccode{Cengine}. 
The actual backend classes corresponding to the three types are \ccode{RscalarB}, \ccode{CscalarB} 
and \ccode{CtensorB}, but these are not directly accessible to the user. 

The following pages list the corresponding operators in function format. For example, the 
command issued to the engine to add a real scalar \m{y} to another real scalar \m{x} and store the 
resulting handle in hdl is 

\texttt{~~~Chandle* hdl=engine.push<rscalar\_add\_op>(xhdl,yhdl)}. 

In the documentation this would appear simply as 

\texttt{~~~rscalar\_add\_op(const rscalar\& x, const rscalar\& y)},

since \ccode{xhdl} and \ccode{yhdl} are \code{Chandle} objects pointing to \ccode{rscalar}s. 
