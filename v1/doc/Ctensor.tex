\begin{cldescr}{\ccode{ctensor}}

The \ccode{ctensor} virtual type represents complex valued matrices and tensors in single precision arithmetic. 
A \ccode{ctensor} object may have a bundle dimension \m{n_{\textrm{bu}}}. 
%The backend storage class for \ccode{ctensor} is \ccode{CtensorB}. 
User level code can access \ccode{ctensor} objects using the following operators. 


\begin{clgroup}[Constructors]
\mmember{
new\_ctensor\_op(const Gdims\& dims, const int nbu=-1, const int dev	=0)\\
new\_ctensor\_zero\_op(const Gdims\& dims, const int nbu=-1, const int dev=0)\\
new\_ctensor\_ones\_op(const Gdims\& dims, const int nbu=-1, const int dev=0)\\
new\_ctensor\_identity\_op(const Gdims\& dims, const int nbu=-1, const int dev=0)\\
new\_ctensor\_sequential\_op(const Gdims\& dims, const int nbu=-1, const int dev=0)\\
new\_ctensor\_gaussian\_op(const Gdims\& dims, const int nbu=-1, const int dev=0)
}{Construct a new \ccode{ctensor} object of size \ccode{dims} with \ccode{nbu} bundles on \ccode{dev}. 
The six cases correspond to the object being (a) uninitialized, (b) initialized to zero, 
(c) the ones tensor, (d) the identity matrix, (e) initilized with entries \m{1,2,\ldots} in sequence, 
(g) initialized with random standard normal entries.  
\ccode{nbu=-1} signifies that the object is not bundled and \ccode{device=0} is the host.}
\\ 
\member{new\_ctensor\_from\_gtensor\_op(const Gtensor\& T, const int nbu=-1, const int dev=0)}{
Create a new \ccode{ctensor} from the \ccode{Gtensor} \ccode{T}.} 
\member{ctensor\_copy\_op(const ctensor\& X)}{Create a new \ccode{ctensor} by copying \ccode{X}.} 
\end{clgroup}

\begin{clgroup}[In-place operators]
\member{ctensor\_set\_zero\_op(const ctensor\& x)}{Set \ccode{x} to zero.}
\end{clgroup}

\begin{clgroup}[Not in-place operators]
\member{ctensor\_conj\_op(const ctensor\& X)}{Return the conjugate tensor \m{\wbar{X}}.}
\member{ctensor\_transp\_op(const ctensor\& X)}{Return \m{X^\top}, the transpose of \m{X}.}
\member{ctensor\_herm\_op(const ctensor\& X)}{Return \m{X^\dag}, the Hermitian conjugate of \m{X}.}
%\member{ctensor\_get\_imag\_op(const ctensor\& x)}{Return the imaginary part of \m{x}.}
\end{clgroup}

\begin{clgroup}[Cumulative operators]
\member{ctensor\_add\_op(const ctensor\& R, const ctensor\& X)}{Set \m{r\leftarrow R+X}.}
\member{ctensor\_add\_conj\_op(const ctensor\& R, const ctensor\& X)}{Set \m{r\leftarrow R+\wbar{X}}.}
\member{ctensor\_add\_transp\_op(const ctensor\& R, const ctensor\& X)}{Set \m{r\leftarrow R+X^\top}.}
\member{ctensor\_add\_herm\_op(const ctensor\& R, const ctensor\& X)}{Set \m{r\leftarrow R+X^\dag}.}
\\ 
\member{ctensor\_add\_times\_real\_op(const ctensor\& R, const ctensor\& X, const float c)}{Set \m{R\leftarrow R+cX}.}
\member{ctensor\_add\_times\_complex\_op(const ctensor\& R, const ctensor\& X, const complex<float> c)}{Set \m{R\leftarrow R+cX}.}
\\ 
\member{ctensor\_add\_prod\_rA\_op(const ctensor\& R, const rscalar\& c, const ctensor\& X)}{Set \m{R\leftarrow R+cX}.}
\member{ctensor\_add\_prod\_cA\_op(const ctensor\& R, const cscalar\& c, const ctensor\& X)}{Set \m{R\leftarrow R+cX}.}
\member{ctensor\_add\_prod\_cc\_A\_op(const ctensor\& R, const cscalar\& c, const ctensor\& X)}{Set \m{R\leftarrow R+\wbar{c}X}.}
\member{ctensor\_add\_prod\_c\_Ac\_op(const ctensor\& R, const cscalar\& c, const ctensor\& X)}{Set \m{R\leftarrow R+c\wbar{X}}.}
\\
\mmember{
ctensor\_add\_Mprod\_op(const ctensor\& R, const ctensor\& A, const ctensor\& B)\\
ctensor\_add\_Mprod\_AT\_op(const ctensor\& R, const ctensor\& A, const ctensor\& B)\\
ctensor\_add\_Mprod\_TA\_op(const ctensor\& R, const ctensor\& A, const ctensor\& B)\\
ctensor\_add\_Mprod\_AC\_op(const ctensor\& R, const ctensor\& A, const ctensor\& B)\\
ctensor\_add\_Mprod\_TC\_op(const ctensor\& R, const ctensor\& A, const ctensor\& B)\\
ctensor\_add\_Mprod\_AH\_op(const ctensor\& R, const ctensor\& A, const ctensor\& B)\\
ctensor\_add\_Mprod\_HA\_op(const ctensor\& R, const ctensor\& A, const ctensor\& B)\\
}{
Set \m{R\leftarrow R+AB},~ \m{R\leftarrow R+AB^\top},~ \m{R\leftarrow R+A^\topB},~ \m{R\leftarrow R+A\wbar{B}},~ 
\m{R\leftarrow R+A^\top\wbar{B}},~ \m{R\leftarrow R+AB^\dag},~  \m{R\leftarrow R+A^\dagB}.}
\\
\member{ctensor\_add\_column\_norms\_op(const ctensor\& R, const ctensor\& X)}{
Increment \m{R(i_1,\ldots,i_{k-1})} with the \m{\ell_2} norm of the column \m{X(i_1,\ldots,i_{k-1},\ts\cdot\ts)}.}
\member{ctensor\_divide\_cols\_op(const ctensor\& R, const ctensor\& N)}{
Divide each element of the column \m{R(i_1,\ldots,i_{k-1},\ts\cdot\ts)} by \m{N(i_1,\ldots,i_{k-1})}.}

\end{clgroup}

\begin{clgroup}[Into operators]
\member{ctensor\_add\_inp\_op(const cscalar\& r, const ctensor\& A, const ctensor\& B)}{
Set \m{r\leftarrow r+\inp{A,B}}.}
\end{clgroup}


\begin{clgroup}[Backward operators]
\item \hspace{-6pt}The following ``backward'' operators are for use in automatic differentiation.
\\ 
\mmember{ctensor\_add\_col\_norms\_back\_op(const ctensor\& R, const ctensor\& G, const ctensor\& X,\hspace{50pt}\mbox{} 
\\ \hfill const ctensor\& N)}{}\vspace{-18pt}
\mmember{ctensor\_add\_divide\_cols\_back0\_op(const ctensor\& R, const ctensor\& G, const ctensor\& X,\hspace{50pt}\mbox{} 
\\ \hfill const ctensor\& N)}{}\vspace{-18pt}
\mmember{ctensor\_add\_divide\_cols\_back1\_op(const ctensor\& R, const ctensor\& G, const ctensor\& X,\hspace{50pt}\mbox{} 
\\ \hfill const ctensor\& N)}{}\vspace{-18pt}
\end{clgroup}

\begin{clgroup}[Blocking operations]
\item \hspace{-6pt}The following functions are called directly as opposed to being pushed to the engine as an operator. 
%and force the engine to flush all operations up to \ccode{x}.
\\ 
\member{Gtensor<complex<float> > ctensor\_get(const ctensor\& X)}{
Flush \ccode{X} and return its value as a \ccode{Gtensor}.}
\end{clgroup}


\end{cldescr}

\ignore{
\member{ctensor\_subtract\_op(const ctensor\& r, const ctensor\& x)}{Set \m{r\leftarrow r-x}.}
\member{ctensor\_add\_prod\_op(const ctensor\& r, const ctensor\& x, const ctensor\& y)}{
	Set \m{r\leftarrow r+xy}.}
\member{ctensor\_add\_div\_op(const ctensor\& r, const ctensor\& x, const ctensor\& y)}{
	Set \m{r\leftarrow r+x/y}.}
\\
\member{ctensor\_add\_to\_real\_op(const ctensor\& r, const rscalar\& x)}{Set \m{r\leftarrow r+(x,0)}.}
\member{ctensor\_add\_to\_imag\_op(const ctensor\& r, const rscalar\& x)}{Set \m{r\leftarrow r+(0,x)}.}
\\
\member{ctensor\_add\_abs\_op(const ctensor\& r, const ctensor\& x)}{Set \m{r\leftarrow r+\abs{x}}.}
\member{ctensor\_add\_exp\_op(const ctensor\& r, const ctensor\& x)}{Set \m{r\leftarrow r+e^{x}}.}
\member{ctensor\_add\_pow\_op(const ctensor\& r, const ctensor\& x, const float p, const float c)}{
Set \m{r\leftarrow r+c\,x^p}.}
}

