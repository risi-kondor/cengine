The most important class in \Cengine{} is that of the compute engine itself, \ccode{Cengine::Cengine}. 
Normally a single instance of this class is initialized at startup and keeps running until 
the user's program terminates. For example, 

\texttt{~~~Cengine::Cengine engine(4)}

initializes a compute engine with 4 CPU threads. 

User level code does not have direct access to the data objects managed by the engine. 
Rather, it can only issues commands to the engine via its \ccode{push} method. 
The \ccode{push} method then returns a handle, which can subsequently be used to reference the object 
stored internally. For example, 

\texttt{~~~Chandle A=engine.push<new\_ctensor\_gaussian>(\{3,3\});}

instructs the engine to create a new \m{3\times 3} complex matrix filled with random numbers 
drawn IID from the standard normal distribution. Issuing the command 

\texttt{~~~Chandle B=engine.push<ctensor\_add>(A,A);}

adds \ccode{A} to itself and stores the result in an object referenced by \ccode{B}. 

\Cengine{} follows the asynchronous, delayed execution model of computation. This means that most  
commands are not executed immediately, but at some later moment in time, when,  
depending on context, either a CPU thread becomes available, or a sufficient number of operations 
of the same type have accumulated for execution on the GPU. 
The order in which the operations are executed might also differ from the order that they 
were issued to the engine. 
However, the engine keeps track of dependencies between operations to ensure that the final result of 
any computation is correct. 
 
%This means that by the time the second of the above commands is issued to the engine, the original 
%matrix \ccode{A} will not have been constructed yet. Therefore the engine enqueues the second operation 
%respecting its dependency on the 

In contrast, a small set of commands are \emph{blocking}, which means that calling function will  
wait until the result has actually been computed. For example, the command 

\texttt{~~~Gtensor M=engine.push<ctensor\_get>(B)}

returns the value of \ccode{B} in a user side tensor object \ccode{M}. This command requires 
explicitly materializing \ccode{B}, therefore it only returns after all computations leading up to \ccode{B} 
are complete and \ccode{B} has been computed as well. Calling 

\texttt{~~~cengine::flush(B)}

has a similar effect, while 

\texttt{~~~cengine::flush()}

flushes all pending operations.  

\Cengine{} also automatically takes care of memory management. In particular, 
for any given backend object \ccode{T}, as soon as there are no operations pending that take \ccode{T} 
as an argument and there are no user side handles pointing to \ccode{T} either, 
\ccode{T} is scheduled for deletion. When the \ccode{Cengine::engine} object is shut down, all pending 
operations are flushed and all backend objects are destroyed. 

\section*{Operators}

Commands in \Cengine{} are actually operators and the template argument of \ccode{Cengine::push} 
command is the name of the corresponding class. The abstract base class of all 
operator classes is \ccode{Cengine::Coperator}.  
For example, the internal definition of the \ccode{ctensor\_add\_op} operator in abbreviated form is 

\boxedcode{
  class ctensor\_add\_op: public Coperator, public CumulativeOperator\{\\
  public:\\ 
\\ 
\phantom{MMM}using Coperator::Coperator; \\ 
\\
\phantom{MMM}void exec()\{\\
\phantom{MMMMMM}owner->obj=inputs[0]->obj;\\
\phantom{MMMMMM}asCtensorB(owner).add(asCtensorB(inputs[1]));\\
\phantom{MMM}\}\\
\\
\phantom{MMM}string str() const\{\\
\phantom{MMMMMM}return "ctensor\_add"+inp\_str();\\
\phantom{MMM}\}\\
\\    
\};
}
\mbox{} 

In each operator class, the \ccode{exec} method is responsible for carrying out the actual operation on 
the operator's arguments. In the case of \ccode{ctensor\_add\_op} this is  
just involves calling the backend object's method for tensor summation. 
However, in user-defined operators, the \ccode{exec} method is often significantly more involved. 
Adding a new operator to \Cengine{} just requires defining the corresponding operator class. 

The concrete data object that \ccode{ctensor\_add\_op} operates on is of type \ccode{Censing::CtensorB}, 
which is the backend container for \ccode{ctensor} objects. 
The built-in \ccode{rscalar, scalar} and \ccode{ctensor}, extended by user defined operators, 
are sufficient for many purposes. However, new backend object classes are also easy to add, 
by subclassing the \ccode{Cengine::Cobject} abstract class. 



 