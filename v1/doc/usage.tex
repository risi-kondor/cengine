\vspace{-10pt}
The most important class in \Cengine{} is the compute engine itself, \ccode{Cengine::Cengine}. 
Normally a single instance of this class is initialized at startup and keeps running until 
the program terminates. For example, 

\texttt{~~~Cengine::Cengine engine(4)}

initializes a compute engine with 4 CPU threads. 

User level code does not have direct access to the data objects managed by the engine. 
Rather, it issues commands to the engine via the engine's \ccode{push} method. 
The \ccode{push} method returns a pointer to a handle, 
which can subsequently be used to reference the object stored internally. For example, 

\texttt{~~~Chandle* A=engine.push<new\_ctensor\_gaussian>(\{3,3\});}

instructs the engine to create a new \m{3\times 3} complex matrix filled with random numbers 
drawn IID from the standard normal distribution. Issuing the command 

\texttt{~~~Chandle* B=engine.push<ctensor\_add>(A,A);}

adds \ccode{A} to itself and stores the result in an object referenced by \ccode{B}. 

%\vspace{-10pt}
\subsubsection*{Asynchronous execution}

\Cengine{} follows the asynchronous, delayed execution model of computation. This means that most  
commands are not executed immediately, but at some later moment in time, when,  
depending on context, either a CPU thread becomes available, or a sufficient number of operations 
of the same type have accumulated for execution on the GPU. 
The order in which the operations are executed need not be the same as the order that they 
were issued to the engine. 
To ensure correctness, the engine needs to keep track of dependencies between operations 
in the form of a directed acyclic graph (DAG). 
% to ensure that the final result is correct. 
 
%This means that by the time the second of the above commands is issued to the engine, the original 
%matrix \ccode{A} will not have been constructed yet. Therefore the engine enqueues the second operation 
%respecting its dependency on the 

Delayed execution implies that the \ccode{Chandle} objects returned by the engine do not 
point to the actual result of the computation, but only to where the result will eventually appear. 
To correctly manage this, user level code will typically encapsulate \ccode{Cengine} calls in 
a separate set of classes. For example, the user might define a \ccode{ComplexMatrix} class which has 
a member variable \ccode{hdl} to store the handle returned by engine. 
To implement in-place matrix addition, the user will add a member function 

\boxedcode{
ComplexMatrix\& ComplexMatrix::operator+(const ComplexMatrix\& B)\{\\
\phantom{MMM}Chandle* t=engine.push<ctensor\_add>(hdl,B.hdl);\\
\phantom{MMM}delete hdl;\\
\phantom{MMM}hdl=t;\\
\}
}

Of course eventually the result of the computation does have to be extracted from the engine. %\ccode{Cengine}. 
For this we use commands that are \emph{blocking}, meaning that the calling function will  
wait until the result has actually been computed. For example, the command 

\texttt{~~~Gtensor M=engine.push<ctensor\_get>(B);}

returns the value of \ccode{B} in a user side tensor object \ccode{M}. This command requires 
explicitly materializing \ccode{B}, therefore it waits until all computations leading up to \ccode{B} 
are complete and \ccode{B} has been computed as well. 

Calling 

\texttt{~~~cengine::flush(B);}

has a similar effect, while 

\texttt{~~~cengine::flush();}

flushes all pending operations.  

\Cengine{} automatically takes care of memory management. 
For any given backend object \ccode{xobj}, when there are no operations pending that take \ccode{xobj} 
as an argument and no user side handles pointing to \ccode{xobj}, 
the object is scheduled for deletion. 
When the \ccode{Cengine} is deleted or shut down, all pending 
operations are flushed and all backend objects are destroyed. 

\section*{Operators}
\addcontentsline{toc}{section}{Operators}

Commands in \Cengine{} are implemented as operators, and each command must have a 
corresponding class. The template argument of \ccode{Cengine::push} 
command is the name of this operator class. The abstract base class of all 
operator classes is \ccode{Cengine::Coperator}.  
For example, the internal definition of the \ccode{ctensor\_add\_op} operator 
(slightly abbreviated) is  

\boxedcode{
  class ctensor\_add\_op: public Coperator, public CumulativeOperator\{\\
  public:\\ 
\\ 
\phantom{MMM}using Coperator::Coperator; \\ 
\\
\phantom{MMM}void exec()\{\\
\phantom{MMMMMM}owner->obj=inputs[0]->obj;\\
\phantom{MMMMMM}asCtensorB(owner).add(asCtensorB(inputs[1]));\\
\phantom{MMM}\}\\
\\
\phantom{MMM}string str() const\{\\
\phantom{MMMMMM}return "ctensor\_add"+inp\_str();\\
\phantom{MMM}\}\\
\\    
\};
}
\mbox{} 

In each operator class, the \ccode{exec} method is responsible for carrying out the actual operation on 
the operator's arguments. In the case of \ccode{ctensor\_add\_op} this just amounts to 
calling the backend object's method for tensor summation. 
However, in user-defined operators the \ccode{exec} method can often be significantly more involved. 
Adding a new operator to \Cengine{} just requires defining the corresponding operator class. 
%and does not require any modification of the engine itself. 

The concrete data object that \ccode{ctensor\_add\_op} operates on is a \ccode{Cengine::CtensorB}, 
which is the backend container for \ccode{ctensor} objects. 
The built-in \ccode{rscalar, scalar} and \ccode{ctensor} classes, 
extended by user defined operators are sufficient for many purposes. 
However, there is nothing stopping the user from adding new backend classes as well, simply 
%new backend object classes are also easy to add, 
by subclassing the \ccode{Cengine::Cobject} abstract class. 
%Adding new backend classes and new operators does not require any 
\Cengine{} can manage any type of user defined backend object, as long as it is derived from 
\ccode{Cengine::Cobject}, and any type of user defined operator, as long as it is derived from 
\ccode{Cengine::Coperator}. 
Adding new objects and operators does not require making any changes to \Cengine{} itself.  

\clearpage
\section*{Batched operators}
\addcontentsline{toc}{section}{Batched operators}

Batching refers to accumulating multiple instances of the same operation and executing them together, in parallel. 
Batching is particularly important efficiently utilizing graphics processor units (GPUs), 
since GPU threads are generally tied: 
on NVIDIA architectures, for example, all threads running on the same streaming 
multiprocessor must essentially be executing the the exact same machine level instruction at any given time. 
Some types of computations, such as 
solving a systems of partial differential equations on a regular grid are well suited to this 
paradigm, since %each thread can take care of a single grid point and 
the operations that need to be performed at each gridpoint are the same.

Other types of computations, however, are much less structured. 
In a graph neural network, for example, the operation performed at each node depends 
on the number of neighbors. In principle, it is possible to write code that 
separately parallelizes over all nodes 
with just one neighbor, all nodes with two neighbors, and so on, 
but in practice such low level multithreading is laborious and highly error prone. 
%at such a low level of detail quickly becomes untenable. 

One solution that has emerged is \emph{dynamic batching}, which refers to accumulating operations of each 
type and executing them together as a batch. %, when a certain number of the same type have accumulated. 
Taking dynamic batching too far can lead to situations where a large number of batched operations 
are mutually waiting on each other and none of the batches are actually run. 
Therefore, as a general principle, it is best to use dynamic batching sparingly, on a 
%in general, it is best to only batch a relatively 
small set of frequent operations that are 
expensive enough to be performance critical, 
yet small enough that executing the operations individually (without batching) would waste much 
of the GPU's parallel processing power.  
%but not so expensive that individually (without batching) 
%a single instance of the operation could itself saturate most of the GPU's processing power. 
Basic matrix operations, such as matrix/scalar and matrix/matrix products are good candidates 
for batching. Accessing individual components of matrices, however, is not an operation that would 
likely benefit from dynamic batching. 

\Cengine{} will attempt to dynamically batch any operator derived from the \ccode{BatchedOperator} class. 
In addition to the \ccode{exec()} method, batched operators must also have a \ccode{batched\_exec} method, 
which takes a \emph{vector} of pointers to compute graph nodes as its argument, and executes 
each node in parallel. For each batched operator class \ccode{UserOp1}, the engine will internally 
create a separate \ccode{BatcherA<UserOp1>} object to manage the batching process. 
%For the engine to be able to 
To keep track of the correspondence between operators and batchers, each batched operator 
class must provide a static integer \ccode{batcher\_id} variable. 

\subsubsection*{Meta-batchers}

Many batched operators require separate batchers for different settings of their parameters. For example, 
in order to batch matrix multiplication, we need separate batchers for each combination of input matrix 
dimensions. \Cengine{} makes it easy to implement such \emph{multi-batched} operators by introducing batcher 
signatures and the \ccode{MetaBatcher} class. 

Any multi-batched class must have a corresponding signature type. For example the signature class of the 
matrix multiplication operator is \ccode{Mprod\_signature}, 
which stores the dimensions of the two matrices to be multiplied and 
some flags to signify if either matrix is transposed. The matrix multiplication operator 
\ccode{ctensor\_Mprod\_op} must have a \ccode{signature()} method that returns the signature object 
corresponding to the given pair of matrices to be multiplied. 
The engine will then create a separate batcher object for each distinct signature. 

The purpose of the \ccode{MetaBatcher} class (for matrix multiplication) is to 
route individual matrix products to their corresponding batcher.  
All that the operator class needs to do enable this process is provide a 
%to define the signature class and provide a 
\ccode{spawn\_batcher()} method that creates the appropriate templated \ccode{MetaBatcher} object. 
In the case of our example the type of this (in slightly abbreviated form) would be 

\texttt{~~~MetaBatcher< ctensor\_add\_Mprod\_op, Mprod\_signature, BatcherA<ctensor\_add\_Mprod\_op> >}.


\section*{Built-in types}
\addcontentsline{toc}{section}{Built-in types}

While \Cengine{} is primarily designed to be used with user-defined data classes and operators, it has a 
a small collection of simple built-in types corresponding to real and complex scalars/tensors to 
seed this process:

\texttt{\phantom{MM}rscalar\\\phantom{MM}cscalar\\\phantom{MM}ctensor
}.

To enable fast GPU computation, each of these classes is implemented in single precision arithmetic (\ccode{float}). 
The corresponding back-end classes are \ccode{RscalarB, CscalarB} and \ccode{CtensorB}. 
Each of these types is equipped with a minimal set of arithmetic and linear algebra operators. 


\subsubsection*{Data layout and GPU functionality}

Similarly to deep learning frameworks such as \ccode{PyTorch} and \ccode{TensorFlow}, \Cengine{}'s built in 
objects can be flexibly moved back and forth between the host and the GPUs. 
This is done by the \ccode{to\_device(d)} command, where \ccode{d} is the identifier of the GPU, or 
\ccode{0} in case that the object is to be moved back to the host. 

In general, every backend operation has two separate implementations: 
one for the execution on the CPU in straight-line \cpp{},  
and once for execution on the GPU written in CUDA or CUBLAS. 
%. Whenever possible that latter is done with CUBLAS calls, but in certain 
%cases it requires custom CUDA kernels. 
Whether a given operation is executed on the CPU or GPU depends on where its arguments reside: in general, 
\Cengine{} will move all input objects to the same device as where the first input argument resides 
and perform the operation on that device.

The storage layout of the built in classes is optimized for GPU computation. In particular, 
matrix/tensor objects are padded to multiples of 32 \ccode{float}s, and complex tensor are stored 
with their real and imaginary parts separate. \Cengine{} uses a row-major matrix/tensor storage format.   