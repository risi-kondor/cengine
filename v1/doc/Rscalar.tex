\begin{cldescr}{\ccode{rscalar}}

The \ccode{rscalar} virtual type is used to represent single precision real valued scalars. 
An \ccode{rscalar} object can store a single real number or a bundle of \m{n_{\textrm{bu}}} real numbers. 
The backend storage class for \ccode{rscalar} is \ccode{RscalarB}. 
User level code can access \ccode{rscalar} objects by pushing one of the following operators to the engine. 

\begin{clgroup}[Constructors]
\mmember{
new\_rscalar\_op(const int nbu=-1, const int dev=0)\\
new\_rscalar\_zero\_op(const int nbu=-1, const int dev=0)\\
new\_rscalar\_set\_op(const int nbu=-1, const float x, const int dev=0)\\
new\_rscalar\_gaussian\_op(const int nbu=-1, const int dev=0)
}{Construct a new \ccode{rscalar} object with bundle size \ccode{nbu} on device \ccode{dev}. 
The four cases correspond to the object being uninitialized, initialized to zero, initialized to \m{x},  
or initialized with random standard normal entries. 
\ccode{nbu=-1} signifies that the object is not bundled and \ccode{dev=0} is the host.}

\member{rscalar\_copy\_op(const rscalar\& x)}{Create a new \ccode{rscalar} by copying \ccode{x}}. 
\end{clgroup}

\begin{clgroup}[In-place operators]
\member{rscalar\_set\_zero\_op(const rscalar\& x)}{Set \ccode{x} to zero.}
\end{clgroup}

\begin{clgroup}[Cumulative operators]
\member{rscalar\_add\_op(const rscalar\& r, const rscalar\& x)}{Set \m{r\leftarrow r+x}.}
\member{rscalar\_subtract\_op(const rscalar\& r, const rscalar\& x)}{Set \m{r\leftarrow r-x}.}
\member{rscalar\_add\_prod\_op(const rscalar\& r, const rscalar\& x, const rscalar\& y)}{
	Set \m{r\leftarrow r+x\ts y}.}
\member{rscalar\_add\_div\_op(const rscalar\& r, const rscalar\& x, const rscalar\& y)}{
	Set \m{r\leftarrow r+x/y}.}
\\
\member{rscalar\_add\_abs\_op(const rscalar\& r, const rscalar\& x)}{Set \m{r\leftarrow r+\abs{x}}.}
\member{rscalar\_add\_exp\_op(const rscalar\& r, const rscalar\& x)}{Set \m{r\leftarrow r+e^{x}}.}
\member{rscalar\_add\_pow\_op(const rscalar\& r, const rscalar\& x, const float p, const float c)}{
Set \m{r\leftarrow r+c\,x^p}.}
\member{rscalar\_add\_ReLU\_op(const rscalar\& r, const rscalar\& x, const float c)}{
Set \m{r\leftarrow r+x}~ if~ \m{x\<\geq 0}, otherwise set \m{r\leftarrow r+c\ts x}.}
\member{rscalar\_add\_sigmoid\_op(const rscalar\& r, const rscalar\& x)}{Set \m{r\leftarrow r+1/(1\<+e^{-x})}.}
\end{clgroup}

\begin{clgroup}[Backward operators]
\item \hspace{-6pt}The following operators are the ``backward'' counterparts of some of the above operators for 
use in automatic differentiation.
\\ 
\mmember{rscalar\_add\_div\_back1\_op(const rscalar\& r, const rscalar\& g, const rscalar\& x,\hspace{60pt}\mbox{}\\
\hfill const rscalar\& y)}{
Set \m{r\leftarrow r-g\,x/y^2}.}
\mmember{rscalar\_add\_pow\_back\_op(const rscalar\& r, const rscalar\& x,  const rscalar\& g, const float p, 
\\\hfill const float c)}{
Set \m{r\leftarrow r+c\,g\,x^p}.}
\member{rscalar\_add\_abs\_back\_op(const rscalar\& r, const rscalar\& g, const rscalar\& x)}{
Set~ \m{r\leftarrow r+g}~ if~ \m{x\<\geq 0} ~and~ \m{r\leftarrow r-g} ~otherwise.}
\member{rscalar\_add\_ReLU\_back\_op(const rscalar\& r, const rscalar\& g, const rscalar\& x, const float c)\hspace{-30pt}\mbox{}}{
Set~ \m{r\leftarrow r+g}~ if~ \m{x\<\geq 0}, ~otherwise~ \m{r\leftarrow r+c\ts g}.}
\member{rscalar\_add\_sigmoid\_back\_op(const rscalar\& r, const rscalar\& g, const rscalar\& x)}{
Set \m{r\leftarrow r+g\ts x/(1\<-x)}.}
\end{clgroup}

\begin{clgroup}[Blocking functions]
\item \hspace{-6pt}The following functions are called directly (as opposed to being pushed to the engine as an operator).
\\ 
\member{vector<float> rscalar\_get(const rscalar\& x)}{
Flush \ccode{x} and return its value(s) in a \ccode{std::vector}. }
\end{clgroup}

\end{cldescr}